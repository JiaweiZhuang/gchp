%                **** IMPORTANT NOTICE *****
% This LaTeX file has been automatically produced by ProTeX v. 1.1
% Any changes made to this file will likely be lost next time
% this file is regenerated from its source. Send questions 
% to Arlindo da Silva, dasilva@gsfc.nasa.gov
 
\setlength{\parskip}{0pt}
\setlength{\parindent}{0pt}
\setlength{\baselineskip}{11pt}
 
%--------------------- SHORT-HAND MACROS ----------------------
\def\bv{\begin{verbatim}}
\def\ev{\end{verbatim}}
\def\be{\begin{equation}}
\def\ee{\end{equation}}
\def\bea{\begin{eqnarray}}
\def\eea{\end{eqnarray}}
\def\bi{\begin{itemize}}
\def\ei{\end{itemize}}
\def\bn{\begin{enumerate}}
\def\en{\end{enumerate}}
\def\bd{\begin{description}}
\def\ed{\end{description}}
\def\({\left (}
\def\){\right )}
\def\[{\left [}
\def\]{\right ]}
\def\<{\left  \langle}
\def\>{\right \rangle}
\def\cI{{\cal I}}
\def\diag{\mathop{\rm diag}}
\def\tr{\mathop{\rm tr}}
%-------------------------------------------------------------

\markboth{Left}{Source File: NUOPC\_Driver.F90,  Date: Mon Mar 28 16:35:43 PDT 2011
}

\bigskip{\sf MODULE:}
\begin{verbatim}  module NUOPC_Driver
\end{verbatim}

\bigskip{\sf DESCRIPTION:\\}
Component that drives and coordinates initialization of its child components: Model, Mediator, and Connector components. For every Driver time step the same run sequence, i.e. sequence of Model, Mediator, and Connector {\tt Run} methods is called. The run sequence is fully customizable. The default run sequence implements explicit time stepping.

\bigskip{\sf SUPER:}
\begin{verbatim}  ESMF_GridComp
\end{verbatim}

\bigskip{\sf USE DEPENDENCIES:}
\begin{verbatim}  use ESMF
\end{verbatim}

\bigskip{\sf SETSERVICES:}
\begin{verbatim}  subroutine SetServices(driver, rc)
    type(ESMF_GridComp)   :: driver
    integer, intent(out)  :: rc
\end{verbatim}

\bigskip{\sf INITIALIZE:}
\begin{itemize}
\item phase 0: ({\sc Required, NUOPC Provided})
  \begin{itemize}
  \item Ensure that the {\tt InitializePhaseMap} and {\tt InternalInitializePhaseMap} attributes are set consistent with the available NUOPC Initialize Phase Definition (IPD) versions (see section \ref{IPD} for a precise definition). The default implementation uses IPDv02 for {\tt InitializePhaseMap}, and sets
    \begin{itemize}
    \item IPDv02p1  ({\sc NUOPC Provided})
    \item IPDv02p3  ({\sc NUOPC Provided})
    \item IPDv02p5  ({\sc NUOPC Provided}).
    \end{itemize}  
 The default implementation uses IPDv05 for {\tt InternalInitializePhaseMap}, and sets
    \begin{itemize}
    \item IPDv05p1  ({\sc NUOPC Provided})
    \item IPDv05p2  ({\sc NUOPC Provided})
    \item IPDv05p3  ({\sc NUOPC Provided})
    \item IPDv05p4  ({\sc NUOPC Provided})
    \item IPDv05p6  ({\sc NUOPC Provided})
    \item IPDv05p8  ({\sc NUOPC Provided}).
    \end{itemize}    
  \end{itemize}  
\item phase 1: ({\sc Required, NUOPC Provided})
  \begin{itemize}
    \item A default Initialize entry point for the higher level (e.g. application level) to initialize the Driver with a single call.
    \item Internally calls into the  {\tt InitializePhaseMap}: IPDv02p1, IPDv02p3, IPDv02p5 phases in sequence.
  \end{itemize}  

\item {\tt InitializePhaseMap}: IPDv02p1 ({\sc NUOPC Provided})
  \begin{itemize}
  \item Allocate and initialize internal data structures.
  \item If the internal clock is not yet set, set the default internal clock to be a copy of the incoming clock, but only if the incoming clock is valid.
  \item {\it Required specialization} to set component services: {\tt label\_SetModelServices}.
  \begin{itemize}
    \item Call {\tt NUOPC\_DriverAddComp()} for all Model, Mediator, and Connector components to be added.
    \item Optionally replace the default clock. 
  \end{itemize}
  \item Create States for all of the child GridComps.
  \item Create Connectors to/from Driver component itself.
  \item Set default run sequence.
  \item Execute Initialize phase=0 for all Model, Mediator, and Connector components. This is the method where each component is required to initialize its {\tt InitializePhaseMap} Attribute.
  \item {\it Optional specialization} to analyze and modify the {\tt InitializePhaseMap} Attribute of the child components before the Driver uses it: {\tt label\_ModifyInitializePhaseMap}.
  \item {\it Optional specialization} to set run sequence: {\tt label\_SetRunSequence}.
  \item Drive the initialize sequence for the child components, compatible with up to IPDv05, as documented in section \ref{IPD}, through IPDv05p3.
  \end{itemize}  

\item {\tt InitializePhaseMap}: IPDv02p3 ({\sc NUOPC Provided})
  \begin{itemize}
  \item Continue to drive the initialize sequence for the child components, compatible with up to IPDv05, as documented in section \ref{IPD}, through IPDv05p7.
  \end{itemize}  

\item {\tt InitializePhaseMap}: IPDv02p5 ({\sc NUOPC Provided})
  \begin{itemize}
  \item Continue to drive the initialize sequence for the child components, compatible with up to IPDv05, as documented in section \ref{IPD}, through IPDv05p8.
  \end{itemize}  

\item {\tt InternalInitializePhaseMap}: IPDv05p1 ({\sc NUOPC Provided})
  \begin{itemize}
  \item Request that fields in export and import State of child components are mirrored onto the driver's own import and export States.
  \item This includes transferring the associated Grid, Mesh, or LocStream objects.
  \end{itemize}  

\item {\tt InternalInitializePhaseMap}: IPDv05p2 ({\sc NUOPC Provided})
  \begin{itemize}
  \item Reset the request of field mirroring.
  \end{itemize}  

\item {\tt InternalInitializePhaseMap}: IPDv05p3 ({\sc NUOPC Provided})
  \begin{itemize}
  \item Add the {\tt REMAPMETHOD=redist} option to all entries in {\tt CplList} attribute on all Connectors to/from the driver itself.
  \item {\it Optional specialization} to modify the {\tt CplList} attribute on all of the Connectors: {\tt label\_ModifyCplLists}.
  \end{itemize}  

\item {\tt InternalInitializePhaseMap}: IPDv05p4 ({\sc NUOPC Provided})
  \begin{itemize}
  \item Check that all connected fields in the driver's own import and export State have a producer connection.
  \end{itemize}  

\item {\tt InternalInitializePhaseMap}: IPDv05p6 ({\sc NUOPC Provided})
  \begin{itemize}
  \item Complete the allocation of all the fields in the driver's own import and export State.
  \end{itemize}  

\item {\tt InternalInitializePhaseMap}: IPDv05p8 ({\sc NUOPC Provided})
  \begin{itemize}
  \item Set the {\tt InitializeDataComplete} consistent with the data-dependency protocol.
  \end{itemize}  

\end{itemize}

\bigskip{\sf RUN:}
\begin{itemize}
\item phase 1: ({\sc Required, NUOPC Provided})
  \begin{itemize}
  \item If the incoming clock is valid, set the internal stop time to one time step interval on the incoming clock.
  \item Drive the time stepping loop, from current time to stop time, incrementing by time step.
  \begin{itemize}
    \item For each time step iteration the Model and Connector components Run() methods are being called according to the run sequence.
  \end{itemize}  
  \end{itemize}    
\end{itemize}

\bigskip{\sf FINALIZE:}
\begin{itemize}
\item phase 1: ({\sc Required, NUOPC Provided})
  \begin{itemize}
  \item Execute the Finalize() methods of all Connector components in order.
  \item Execute the Finalize() methods of all Model components in order.
  \item {\it Optional specialization} to finalize custom parts of the component: {\tt label\_Finalize}.
  \item Destroy all Model components and their import and export states.
  \item Destroy all Connector components.
  \item Internal clean-up.
  \end{itemize}      
\end{itemize}

\mbox{}\hrulefill\ 

%...............................................................

