% $Id$

\section{How to Adapt Applications for ESMF}
\label{sec:Adoption}

In this section we describe how to bring existing applications 
into the framework.

\subsection{Individual Components}

\begin{itemize}

\item Decide what parts will become Gridded Components 

A Gridded Component is a self-contained
piece of code which will be initialized, will be called once or many times
to run, and then will be finalized.  It will be expected to either take in
data from other components/models, produce data, or both.

Generally a computational model like an ocean or atmosphere model will
map either to a single component or to a set of multiple nested
components.

\item Decide what data is produced 

A component provides data to other components using an ESMF State
object.  A component should fill the State object with a description of
all possible values that it can export.  Generally, a piece of code
external to the component (the AppDriver, or a parent component) will 
be responsible for marking which of these items are actually going to
be needed.  Then the component can choose to either produce all possible
data items (simpler but less efficient) or only produce the data items
marked as being needed.  The component should consult the \htmladdnormallink
{CF data naming conventions}{http://cf-pcmdi.llnl.gov/} when it is listing
what data it can produce.

\item Decide what data is needed 

A component gets data from other components using an ESMF State object.
The application developer must figure out how to get any required 
fields from other components in the application.

\item Make the data blocks private 

A component should communicate to other components only through the
framework.  All global data items should be private to Fortran modules,
and ideally should be isolated to a single derived type which is allocated
at run time.   

\item Divide the code up into start/middle/end phases 

A component needs to provide 3 routines which handle
initialization, running, and finalization.  (For codes which have
multiple phases of initialize, run, and finalize it is possible to have
multiple initialize, run, and finalize routines.)

The initialize routine needs to allocate space, initialize
data items, boundary conditions, and do whatever else is necessary in
order to prepare the component to run.

For a sequential application in which all components are on the same
set of processors, the run phase will
be called multiple times.  Each time the model is expected to take in
any new data from other models, do its computation, and produce data
needed by other components.   A concurrent model, in which different
components are run on different processors, may execute the same 
way.  Alternatively, it may have its run routine called
only once and may use different parts of the framework to arrange
data exchange with other models.  This feature is not yet implemented
in ESMF.

The finalize routine needs to release space, write out results,
close open files, and generally close down the computation gracefully.

\item Make a "Set Services" subroutine 

Components need to provide only a single externally visible entry point.
It will be called at start time, and its job is to register with the
framework which routines satisfy the initialize, run, and finalize
requirements.  If it has a single derived type that holds its private data,
that can be registered too.

\item Create ESMF Fields and FieldBundles for holding data

An ESMF State object is fundamentally an annotated list of other
ESMF items, most often expected to be ESMF FieldBundles (groups of
Fields on the same grid).  Other things which can be placed in a 
State object are Fields, Arrays (raw data with no gridding/coordinate 
information)
and other States (generally used by coupling code).  Any data which is
going to be received from other components or sent to other components
needs to be represented as an ESMF object.

To create an ESMF Field the code must create an ESMF Array object to
contain the data values, and usually an ESMF Grid object to describe the
computational grid where the values are located.  If this is an
observational data stream the locations of the data values will be held in
an ESMF Location Stream object instead of a Grid. 

\item Be able to read an ESMF clock

During the execution of the run routine, information about time 
is transferred between components through ESMF Clocks.  The 
component needs to be able to at least query a Clock for the 
current time using framework methods.

\item Decide how much of the lower level infrastructure to use

The ESMF framework provides a rich set of time management functions,
data management and query functions, and other utility
routines which help to insulate the user's code from the differences
in hardware architectures, system software, and runtime environments.
It is up to the user to select which parts of these functions they
choose to use.

\end{itemize}

\subsection{Full Application}

\begin{itemize}

\item Decide on which components to use 

Select from the set of ESMF components available.

\item Understand the data flow in order to customize a Coupler Component

Examine what data is produced by each component and what data is
needed by each component.  The role of Coupler Components in the
ESMF is to set up any necessary regridding and data conversions
to match output data from one component to input data in another.

\item Write or adapt a Coupler Component

Decide on a strategy for how to do the coupling.  There can be a single
coupler for the application or multiple couplers.
Single couplers follow a "hub and
spoke" model.
Multiple couplers can couple between subsets of the components, and
can be written to couple either only one-way
(e.g. output of component A into input of component B), or two-way
(both A to B and B to A).  

The coupler must understand States, Fields, FieldBundles, 
Grids, and Arrays and ESMF execution/environment objects
such as DELayouts.    

\item Use or adapt a main program 

The main program can be a copy of a driver found in any of the
{\tt system\_tests} sub-directories.  The customization needed is to
{\tt use} the correct Component module files, to gain access to the
{\tt SetServices} routines. 

Although ESMF provides example source code for the main program, it is
{\bf not} considered part of the framework and can be changed by
the user as needed.

The final thing the main program must do is call {\tt ESMF\_Finalize()}.
This will close down the framework and release any associated resources.

The main program is responsible for creating a top-level
Gridded Component, which in turn creates other Gridded and Coupler 
Components.  We encourage this hierarchical design because it
aids in extensibility - the top level Gridded Component can be
nested in another larger application.
The top-level component contains the main time loop and is 
responsible for calling the
{\tt SetServices} entry point for each child component it creates.

\end{itemize}



